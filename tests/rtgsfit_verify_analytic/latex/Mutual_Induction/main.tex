\documentclass{article}
\usepackage[a4paper, margin=1in]{geometry}
\usepackage{amsmath}
\usepackage{hyperref}

\title{Mutual Induction Infinite Wire}
\author{Alexander Prokopyszyn}
\date{June 2025}

\begin{document}

\maketitle

\section{Introduction}

We define $A(x,y)$ as the $z$-component of the magnetic vector potential which satisfies
\[\frac{\partial A}{\partial y} = B_x(x, y),\]
\[\frac{\partial A}{\partial x} = -B_y(x, y),\]
and we choose our integration constant to ensure
\[A(0,0)=0.\]

The mutual induction $M_i(x, y, x_i, y_i)$ enables us to calculate the $A_i$ generated at $(x, y)$ from an infinitely long vertical wire with current, $I_i$, at $(x_i, y_i)$, i.e.
\[A_i(x,y) = M(x, y, x_i, y_i)I_i.\]
The total $A$ generated by multiple filaments is given by
\[A(x,y) = \sum_{i} M(x, y, x_i, y_i)I_i.\]

\section{Mutual induction formula}

\[\begin{aligned}
M(x, y, x_i, y_i) = -\frac{\mu_0}{2\pi} \ln\left(\frac{\rho_i}{d_i}\right) \\
\end{aligned}\]
where
\[\rho_i = \sqrt{(x-x_i)^2+(y-y_i)^2},\]
\[d_i = \sqrt{x_i^2 + y_i^2}.\]
Note that we divide by $d_i$ to ensure $A(0,0)=0$.

\section{Check Ampere's law is satisfied}

We can check Ampere's law is satisfied by switching to cylindrical coordinates with our origin at $(x_i, y_i)$. Hence,
\[A_i(x, y) = A_i(\rho_i) =  -\frac{\mu_0I_i}{2\pi} \ln\left(\frac{\rho_i}{d_i}\right)\]
\[\begin{aligned}
B_{\phi,i}(\rho_i) &= -\frac{\partial A_i}{\partial \rho_i}, \\
&= \frac{\mu_0I_i}{2\pi\rho_i}.
\end{aligned}\]
Hence, we can clearly see that Ampere's law is satisfied.

\section{Resolving the singularity}

$M(x, y, x_i, y_i)$ has a singularity at $(x,y)=(x_i, y_i)$. Strictly speaking, there is no way to resolve this singularity. However, if we wish to know $A$ at $(x_i, y_i)$, we can instead calculate the average value of $A$ over a box with vertices
$\left(x_i \pm \frac{\Delta x}{2}, y_i\pm \frac{\Delta y}{2}\right)$, which is given by
\[\begin{aligned}
\langle A_i\rangle 
&= \int_{y_i-\Delta y /2}^{y_i+\Delta y / 2} \int_{x_i-\Delta x/2}^{x_i+\Delta x / 2}  \frac{M(x, y, x_i, y_i)I_i}{\Delta x\Delta y} dxdy \\
&= I_i \langle M(x_i, y_i) \rangle
\end{aligned}\]

\subsection{Calculating average value of the self-inductance}

Let
\[x'=(x-x_i) / d_i,\]
\[y'= (y-y_i) / d_i,\]
\[\Delta x' = \frac{\Delta x}{d_i},\]
\[\Delta y' = \frac{\Delta y}{d_i}.\]
Hence,
\[\langle M(x_i, y_i) \rangle = -\frac{\mu_0 d_i^2}{4\pi\Delta x\Delta y}\int_{-\Delta y /(2d_i)}^{\Delta y / (2d_i)} \int_{-\Delta x/(2d_i)}^{\Delta x / (2d_i)}  \ln(x'^2+y'^2) dx'dy'.\]
Note that
\[\int_{-\Delta x/(2d_i)}^{\Delta x / (2d_i)}  \ln(x'^2+y'^2) dx' = \frac{\Delta x}{d_i}\ln\left(\frac{\Delta x^2}{4d_i^2} + y'^2\right) + 4y'\tan^{-1}\left(\frac{\Delta x / (2d_i)}{y'}\right) - \frac{2\Delta x}{d_i}.\]
Now we will integrate each term of the above equation one by one
\[\int_{-\Delta y /(2d_i)}^{\Delta y / (2d_i)} \frac{\Delta x}{d_i}\ln\left(\frac{\Delta x^2}{4d_i^2} + y'^2\right) dy' = \frac{\Delta x \Delta y}{d_i^2}\ln\left(\frac{\Delta x^2}{4d_i^2} + \frac{\Delta y^2}{4d_i^2}\right) + 2\frac{\Delta x^2}{d_i^2}\tan^{-1}\left(\frac{\Delta y}{\Delta x}\right) - 2\frac{\Delta x \Delta y}{d_i^2},\]
% \[\int_{-\Delta y /(2d_i)}^{\Delta y / (2d_i)} 4y'\tan^{-1}\left(\frac{\Delta x / (2d_i)}{y'}\right) dy' = \left(\frac{\Delta x^2 + \Delta y^2}{d_i^2}\right)\tan^{-1}\left(\frac{\Delta x}{\Delta y}\right)+\frac{\Delta x \Delta y}{d_i^2} - \frac{\Delta x^2\Delta y}{4d_i^3},\]
\[\int_{-\Delta y /(2d_i)}^{\Delta y / (2d_i)} 4y'\tan^{-1}\left(\frac{\Delta x / (2d_i)}{y'}\right) dy' = \frac{\Delta y^2}{d_i^2}\tan^{-1}\left(\frac{\Delta x}{\Delta y}\right)-\frac{\Delta x^2}{d_i^2}\tan^{-1}\left(\frac{\Delta y}{\Delta x}\right) + \frac{\Delta x \Delta y}{d_i^2},\]
\[\int_{-\Delta y /(2d_i)}^{\Delta y / (2d_i)} - 2\frac{\Delta x}{d_i^2} dy' = - 2\frac{\Delta x \Delta y}{d_i^2}.\]
Hence,
\[\begin{aligned}
\int_{-\Delta y /(2d_i)}^{\Delta y / (2d_i)}\int_{-\Delta x/(2d_i)}^{\Delta x / (2d_i)}  \ln(x'^2+y'^2) dx'dy' &= \frac{\Delta x \Delta y}{d_i^2}\ln\left(\frac{\Delta x^2}{4d_i^2} + \frac{\Delta y^2}{4d_i^2}\right) + \frac{\Delta x^2}{d_i^2}\tan^{-1}\left(\frac{\Delta y}{\Delta x}\right) \\
&+  \frac{\Delta y^2}{d_i^2}\tan^{-1}\left(\frac{\Delta x}{\Delta y}\right) - 3\frac{\Delta x \Delta y}{d_i^2}
\end{aligned}\]
We verify these integrals here:
\url{https://www.desmos.com/calculator/tiegn2skei}.

Hence,
\[
\boxed{\begin{aligned}
\langle M(x_i, y_i) \rangle  &= -\frac{\mu_0}{4\pi}\left[\ln\left(\frac{\Delta x^2+\Delta y^2}{4d_i^2}\right) + \frac{\Delta x}{\Delta y}\tan^{-1}\left(\frac{\Delta y}{\Delta x}\right) 
+  \frac{\Delta y}{\Delta x}\tan^{-1}\left(\frac{\Delta x}{\Delta y}\right) - 3\right]
\end{aligned}}\]

\subsection{Taking limits of the self-inductance}

Note that
\[\lim_{\Delta x\rightarrow0}\left\{\frac{\Delta y}{\Delta x}\tan^{-1}\left(\frac{\Delta x}{\Delta y}\right)\right\} = 1,\]
\[\lim_{\Delta x\rightarrow0}\left\{\frac{\Delta x}{\Delta y}\tan^{-1}\left(\frac{\Delta y}{\Delta x}\right)\right\} = 0.\]
Hence,
\[\begin{aligned}
\lim_{\Delta x\rightarrow0}\langle M(x_i, y_i) \rangle &= -\frac{\mu_0}{4\pi}\left[\ln\left(\frac{\Delta y^2}{4d_i^2}\right) - 2\right], \\
&= -\frac{\mu_0}{2\pi}\left[\ln\left(\frac{\Delta y}{2d_i}\right) - 1\right].
\end{aligned}\]
\[\begin{aligned}
\lim_{\Delta y\rightarrow0}\langle M(x_i, y_i) \rangle = -\frac{\mu_0}{2\pi}\left[\ln\left(\frac{\Delta x}{2d_i}\right) - 1\right].
\end{aligned}\]
These last two equations give the self-induction average induction along a line.


\end{document}
